%%%%%%%%%%%%%%%%%%%%%%%%%%%%%%%%%%%%%%%%%%%%%%%%%%%%%%%%%%%%%%%%%%%%%%%%%%%%%%
%%%%%%%%%%%%%%%%%%%%%%%%%%%%%%%%%%%%%%%%%%%%%%%%%%%%%%%%%%%%%%%%%%%%%%%%%%%%%%
%%
%% Dokumentace k interpretu jazyka IFJ14, 2014
%%
%% Upravená původní dokumentace od Davida Martinka.
%%%%%%%%%%%%%%%%%%%%%%%%%%%%%%%%%%%%%%%%%%%%%%%%%%%%%%%%%%%%%%%%%%%%%%%%%%%%%%
%%%%%%%%%%%%%%%%%%%%%%%%%%%%%%%%%%%%%%%%%%%%%%%%%%%%%%%%%%%%%%%%%%%%%%%%%%%%%%
\documentclass[12pt,a4paper,titlepage,final]{article}

% cestina a fonty
\usepackage[slovak]{babel}
\usepackage[utf8]{inputenc}
% balicky pro odkazy
\usepackage[bookmarksopen,colorlinks,plainpages=false,urlcolor=blue,unicode]{hyperref}
\usepackage{url}
\usepackage{amsmath}
\usepackage{capt-of}
\usepackage[Q=yes]{examplep}
\usepackage{enumitem}

% obrazky
\usepackage{graphicx}
% velikost stranky
\usepackage[text={15.2cm, 25cm}, ignorefoot]{geometry}



\begin{document}

%%%%%%%%%%%%%%%%%%%%%%%%%%%%%%%%%%%%%%%%%%%%%%%%%%%%%%%%%%%%%%%%%%%%%%%%%%%%%%
% titulní strana
\begin{titlepage}

\vspace*{2cm}
\begin{figure}[!h]
  \centering
  \includegraphics[width=10cm]{img/logo.eps}
\end{figure}

\vfill

\begin{center}

\begin{Huge}
Nástroj monitorovania RIP a RIPng
\end{Huge}

\bigskip

\begin{Large}
Sieťové aplikácie a správa sietí
\end{Large}

\bigskip

\begin{Large}
	2015/2016
\end{Large}

\end{center}

\vfill

\begin{center}
\begin{Large}
\today
\end{Large}
\end{center}

\vfill

\begin{flushleft}
\begin{normalsize}
\begin{tabular}{ll}
\bf Autor:\hspace{3mm} & Peter Gazdík (\url{xgazdi03@stud.fit.vutbr.cz}) \\[3mm]

& Fakulta Informačních Technologií \\
& Vysoké Učení Technické v~Brně
\end{tabular}
\end{normalsize}
\end{flushleft}
\end{titlepage}


%%%%%%%%%%%%%%%%%%%%%%%%%%%%%%%%%%%%%%%%%%%%%%%%%%%%%%%%%%%%%%%%%%%%%%%%%%%%%%
% obsah
\pagestyle{plain}
\pagenumbering{roman}
\setcounter{page}{1}
\tableofcontents

%%%%%%%%%%%%%%%%%%%%%%%%%%%%%%%%%%%%%%%%%%%%%%%%%%%%%%%%%%%%%%%%%%%%%%%%%%%%%%
% textova zprava
\newpage
\pagestyle{plain}
\pagenumbering{arabic}
\setcounter{page}{1}

%%%%%%%%%%%%%%%%%%%%%%%%%%%%%%%%%%%%%%%%%%%%%%%%%%%%%%%%%%%%%%%%%%%%%%%%%%%%%%
\section{Popis implementácie} 
%%%%%%%%%%%%%%%%%%%%%%%%%%%%%%%%%%%%%%%%%%%%%%%%%%%%%%%%%%%%%%%%%%%%%%%%%%%%%%

\subsection{Sniffer RIPv1, RIPv2 a RIPng správ}

Implementácia využíva knižnice `libpcap`, ktorá umožňuje odchytávanie paketov v promiskuitnom móde sieťovej karty.

Aby klientská aplikácia nemusela spracovávať všetku komunikácia prichádzajúcu na sieťovú kartu, je využité filtrovanie, ktoré poskytuje táto knižica a ktoré obsluhuje jadro operačného systému. Nedochádza tak k častému prepínaniu kontextu. Takýmto spôsobom je zabezpečené, že aplikácia spracováva len komunikáciu, ktorá prichádza na UDP port číslo 520 a 521, viď popis jednotlivých protokolov.

Po prijatí paketu dochádza k jeho postupnému \uv{vybaľovaniu}. Ako prvá sa na začiatku správy nachádza ethernetová hlavička, Nakoľko nás ale jej obsah nezaujíma a zároveň má fixnú dĺžku, môžeme sa posunúť priamo na IP hlavičku.

V prvom kroku musíme naskôr určiť verziu IP protokolu, t.j. či sa jedná o protokol IPv4 alebo IPv6. Po určení správnej verzie môžeme určiť odosieľateľa správy, ktorého uvedieme neskôr vo výpise. Následne sa môžeme v prípade IPv6 posunúť o fixnú dĺžku a v prípade IPv4 o veľkosť hlavičky, ktorá sa uvádza hneď za verziou protokolu.

Posledným protokolom obaľujúci RIP správu je UDP protokol, z ktorého môžeme pohodlne určiť veľkosť samotnej RIP správy a započať tak jej spracovávanie.

Spracovávanie správy vykonávame s prihliadnutím na verziu smerovacieho protokolu. Medzi protokolom RIPv1 a RIPv2 nie sú až také veľké rozdiely, avšak protokol RIPng je značne odlišný. Napriek tomu, že protokol RIPng nesie záznamy o IPv6 adresách, položky s týmito záznammi majú rovnakú dĺžku ako u RIP protokolu.

Výstup z aplikácie je možné vidieť v kapitole \ref{utok}.

\subsection{Podvrhovač falošných RIPv2 Response správ}

Po spracovaní všetkých parametrov, ktorých je v prípade tejto aplikácie neúrekom, dochádza k zostaveniu RIPv2 paketu.

Po jeho zostavení, ktoré nie je nijak záludné, je odoslaný využitím BSD soketov. Nakoľko potrebujeme odosielať z UDP portu číslo 520, je potrebné aplikáciu spúšťať s oprávnením roota. 

Ak nie je špecifikované rozhranie, na ktoré má byť paket odoslaný, je odoslaný na všetky dostupné rozhrania systému.

\newpage

%%%%%%%%%%%%%%%%%%%%%%%%%%%%%%%%%%%%%%%%%%%%%%%%%%%%%%%%%%%%%%%%%%%%%%%%%%%%%%
\section{Popis útoku} \label{utok}
%%%%%%%%%%%%%%%%%%%%%%%%%%%%%%%%%%%%%%%%%%%%%%%%%%%%%%%%%%%%%%%%%%%%%%%%%%%%%%

Využitím aplikácie \texttt{myripsniffer} získame detailnejšie informácie o prebiehajúcej komunikácii na rozhraní, na ktorom chceme vykonať útok. V tomto prípade budeme zachytávať komunikáciu na rozhraní \texttt{eth0}. Aplikáciu je nutné spustíť s právami roota z dôvodu, že využíva `libpcap` knižnicu.

\begin{verbatim}
sudo ./myripsniffer -i eth0
\end{verbatim}

Aplikácia je schopná zachytiť tri druhy protokolov: RIPv1, RIPv2 a RIPng. Pri všetkých prípadoch sa môže jednať o správy typu \texttt{Request} alebo \texttt{Response}. Vo verzii 2 protokolu RIP je možné vidieť okrem záznamov aj zabezpečenie správy spolu s heslom.

Zachytená komunikácia vyzerá napr. následovne: 

\begin{verbatim}
[23:04:03] RIPv2 from 10.0.0.1
Command: Request (1)
========== ENTRY ===========
Route Tag: 0
Address Family Identifier: unknown (0)
Metric: 16

[23:04:04] RIPv2 from 10.0.0.1
Command: Response (2)
====== AUTHENTICATION ======
Authentication type: Simple password
Password: ISA>28114bb8715
========== ENTRY ===========
Route Tag: 0
Address Family Identifier: IP (2)
IP Address: 10.48.51.0
Netmask: 255.255.255.0
Next Hop: 0.0.0.0
Metric: 1
========== ENTRY ===========
		    ...

[23:04:04] RIPng from fe80::a00:27ff:fe1b:716d
Command: Response (2)
========== ENTRY ===========
Route Tag: 0
IPv6 Prefix: fd00::
Prefix Length: 64
Metric: 1
========== ENTRY ===========
			...
\end{verbatim}

Najdôležitejšia časť pre úspešné vykonanie útoku je záznam s heslom, ktorým musíme náš záznam podpísať.

K podvrhnutiu falošného záznamu využijeme aplikáciu \texttt{myripresponse}. Útok môže vyzerať napr. následovne:

\begin{verbatim}
sudo ./myripresponse -i eth0 -r 10.10.10.0/24 -p "ISA>28114bb8715"
\end{verbatim}

Aj v tomto prípade je nutné spustenie s oprávneniami roota, nakoľko aplikácia využíva port číslo 520. Ak všetko prebehlo úspešne, v smerovacej tabuľke routera sa bude nachádzať nový záznam obshujúci cestu k sieti \texttt{10.10.10.0/24}.

\newpage

\bibliographystyle{czechiso}

\begin{flushleft}
	\bibliography{literatura}
\end{flushleft}



\end{document}
